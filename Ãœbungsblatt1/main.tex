\documentclass[12pt, letterpaper]{article}
\usepackage{amsmath} % Required for mathematical symbols
\usepackage{amssymb} % Required for more mathematical symbols
\usepackage{amsthm}
\usepackage{graphicx} % Required for inserting images
\usepackage{enumitem} % For smaller bullet points
\usepackage{geometry}
 \geometry{
 a4paper,
 total={170mm,257mm},
 left=20mm,
 top=20mm,
 }

\title{Analysis 1 Übungsblatt 1}
\author{Jarne, Lars}
\date{}

\begin{document}

\maketitle 

\paragraph{Aufgabe 1}

\noindent Beweisen Sie die folgenden Äquivalenzen

\paragraph{(a)} $\neg (A \land B) \text{ ist äquivalent zu } \neg A \lor \neg B.$\\

\[
\begin{array}{|c|c|c|c|c|c|c|}
\hline
A & B & A \land B & \neg A & \neg B & \neg A \lor \neg B & \neg(A \land B) \\
\hline
1 & 1 & 1 & 0 & 0 & 0 & 0 \\
1 & 0 & 0 & 0 & 1 & 1 & 1 \\
0 & 1 & 0 & 1 & 0 & 1 & 1 \\
0 & 0 & 0 & 1 & 1 & 1 & 1 \\
\hline
\end{array}
\]\\

\paragraph{(b)} $A \Rightarrow B$ ist äquivalent zu $\neg B \Rightarrow \neg A$\\

\[
\begin{array}{|c|c|c|c|c|c|}
\hline
A & B & \neg A & \neg B & A \Rightarrow B & \neg B \Rightarrow \neg A \\
\hline
1 & 1 & 0 & 0 & 1 & 1 \\
1 & 0 & 0 & 1 & 0 & 0 \\
0 & 1 & 1 & 0 & 1 & 1 \\
0 & 0 & 1 & 1 & 1 & 1 \\
\hline
\end{array}
\]\\

\paragraph{(c)} $A \lor (B \land C)$ ist äquivalent zu $(A \lor B) \land (A \lor C)$\\

\[
\begin{array}{|c|c|c|c|c|c|c|c|}
\hline
A & B & C & B \land C & A \lor B & A \lor C & A \lor (B \land C) & (A \lor B) \land (A \lor C) \\
\hline
1 & 1 & 1 & 1 & 1 & 1 & 1 & 1 \\
1 & 1 & 0 & 0 & 1 & 1 & 1 & 1 \\
1 & 0 & 1 & 0 & 1 & 1 & 1 & 1 \\
1 & 0 & 0 & 0 & 1 & 1 & 1 & 1 \\
0 & 1 & 1 & 1 & 1 & 1 & 1 & 1 \\
0 & 1 & 0 & 0 & 1 & 0 & 0 & 0 \\
0 & 0 & 1 & 0 & 0 & 1 & 0 & 0 \\
0 & 0 & 0 & 0 & 0 & 0 & 0 & 0 \\
\hline
\end{array}
\]\\ \\ \\






\paragraph{Aufgabe 2}

\noindent Sei $M$ eine Menge. Für jedes Element $x \in M$ bezeichne $A(x)$ eine gegebene Aussage. Zeigen Sie:

\paragraph{(a)} $\quad \neg \left( \forall x \in M : A(x) \right) \Leftrightarrow \exists x \in M : \neg A(x).$

\begin{enumerate}
    \item[]{``$\Rightarrow$'':}
            $\neg \left( \forall x \in M : A(x) \right)$ bedeutet: Es ist nicht wahr, dass für alle $x$ aus $M$ die Aussage gilt.
            Diese Aussage ist wahr, wenn es ein $x \in M$ gibt, für welches $A(x)$ nicht gilt, also: $\exists x \in M : \neg A(x)$.
    \item[]{``$\Leftarrow$'':}
            $\exists x \in M : \neg A(x)$ bedeutet: Es existiert ein $x$ aus $M$ für das $A(x)$ nicht gilt. Das bedeutet, dass es nicht für alle $x$ aus $M$ gelten kann, also: $\neg \left( \forall x \in M : A(x) \right)$.
\end{enumerate}

\paragraph{(b)} $\quad \neg \left( \exists x \in M : A(x) \right) \Leftrightarrow \forall x \in M : \neg A(x).$

\begin{enumerate}
    \item[]{``$\Rightarrow$'':}
            $\neg \left( \exists x \in M : A(x) \right)$ bedeutet: Es ist nicht wahr, dass es ein $x$ aus $M$ gibt, für das $A(x)$ gilt.
            Also bedeutet das, dass für jedes $x$ aus $M$ die Aussage $A(x)$ nicht gilt: $\forall x \in M : \neg A(x)$.
    \item[]{``$\Leftarrow$'':}
           $\forall x \in M : \neg A(x)$ bedeutet: Für alle $x$ aus $M$ gilt, dass $A(x)$ nicht gilt. Also gibt es kein $x$ aus $M$, für das $A(x)$ gilt: $\neg \left( \exists x \in M : A(x) \right)$.
\end{enumerate}





\paragraph{Aufgabe 3}

\paragraph{(a)} Gegeben seien die folgenden Mengen:

\begin{enumerate}
    \item[$X =$] $\{n \in \mathbb{N} \mid 1 \leq n \leq 100\},$
    \item[$A =$] $\{n \in X \mid 2(n - 13)(n - 3) < 0\},$
    \item[$B =$] $\{n \in X \mid \text{es gibt ein } m \in \mathbb{N} \text{ mit } m^2 = n\},$
    \item[$C =$] $\{n \in X \mid n \text{ ist durch 2 teilbar}\}.$
\end{enumerate}\\

\noindent Bestimmen Sie die Mengen:

\begin{enumerate}
    \item $A \cup B$ - $C$ = \{1, 5, 7, 9, 11, 25, 49, 81\}
    \item $A \cup (B - C)$ = \{1, 4, 5, 6, 7, 8, 9, 10, 11, 12, 25, 49, 81\}
    \item $(B \cap A)$ - C = \{9\}
    \item $B \cap (A$ - C) = \{9\}
\end{enumerate}

\paragraph{(b)} Seien $X, Y, Z$ Mengen. Beweisen Sie die De Morganschen Regeln:

\paragraph{(i)} $X - (Y \cap Z)$ = $(X - Y) \cup (X - Z)$

\begin{enumerate}
    \item[]{``$\Rightarrow$'':} Sei $x \in X - (Y \cap Z)$.
        \begin{enumerate}[label=$\circ$]
            \item Dann ist $x \in X \land x \notin (Y \cap Z)$.
            \item Daraus folgt, dass $(x \notin Y) \lor (x \notin Z)$, da $x$ sonst im Schnitt wäre.
            \item Wenn $x \notin Y$, dann $x \in X - Y$.
            \item Wenn $x \notin Z$, dann $x \in X - Z$.
            \item Aus Punkt 2 folgt, dass $x$ mindestens in $(X - Y) \lor (X - Z)$.
            \item Somit folgt $x \in (X - Y) \cup (X - Z)$.
        \end{enumerate}
    \item[]{``$\Leftarrow$'':} Sei $x \in (X - Y) \cup (X - Z)$.
        \begin{enumerate}[label=$\circ$]
            \item Somit ist $x \in X \land (x \in (X- Y) \lor x \in (X - Z))$.
            \item Daraus folgt $x \notin Y \lor x \notin Z$.
            \item Da Punkt 2 gilt, muss $x \notin (Y \cap Z)$, da sonst $x \in Y \land x \in Z$ wäre.
            \item Somit ist $x \in X \land x \notin (Y \cap Z)$.
            \item Daraus folgt $x \in X - (Y \cap Z)$.
        \end{enumerate}
\end{enumerate}

\noindent Da beide Implikationen gezeigt wurden, folgt: $X - (Y \cap Z)$ = $(X - Y) \cup (X - Z)$.

\paragraph{(ii)} $X - (Y \cup Z)$ = $(X - Y) \cap (X - Z)$

\begin{enumerate}
    \item[]{``$\Rightarrow$'':} Sei $x \in X - (Y \cup Z)$.
        \begin{enumerate}[label=$\circ$]
            \item Dann ist $x \in X \land x \notin (Y \cup Z)$.
            \item Daraus folgt, dass $x \notin Y \land x \notin Z$, da $x$ nicht in der Vereinigung von $Y$ und $Z$ ist.
            \item Wenn $x \notin Y$, dann $x \in X - Y$.
            \item Wenn $x \notin Z$, dann $x \in X - Z$.
            \item Da beide Bedingungen erfüllt sind, folgt $x \in (X - Y) \land x \in (X - Z)$.
            \item Somit folgt $x \in (X - Y) \cap (X - Z)$.
        \end{enumerate}
    \item[]{``$\Leftarrow$'':} Sei $x \in (X - Y) \cap (X - Z)$.
        \begin{enumerate}[label=$\circ$]
            \item Dann ist $x \in X \land x \in (X - Y) \land x \in (X - Z)$.
            \item Das bedeutet, dass $x \notin Y \land x \notin Z$.
            \item Daraus folgt, dass $x \notin (Y \cup Z)$, da $x$ weder in $Y$ noch in $Z$ enthalten ist.
            \item Da $x \in X$ und $x \notin (Y \cup Z)$, folgt $x \in X - (Y \cup Z)$.
        \end{enumerate}
\end{enumerate}

\noindent Da beide Implikationen gezeigt wurden, folgt: $X - (Y \cup Z) = (X - Y) \cap (X - Z)$. \\







\paragraph{Aufgabe 4}

\noindent Seien $X, Y$ Mengen und $f : X \to Y$ eine Abbildung.

\begin{enumerate}
    \item[(i)] Für $A \subseteq X$ setzen wir 
        $f(A) := \{f(a) \mid a \in A\}$.
    \item[(ii)] Für $B \subseteq Y$ setzen wir 
        $f^{-1}(B) := \{x \in X \mid f(x) \in B\}$.
\end{enumerate}

\noindent Welche der folgenden Aussagen sind wahr? Begründen oder widerlegen Sie.\\

\paragraph{(a)} Für alle $A, B \subseteq Y$ gilt $f^{-1}(A \cap B) = f^{-1}(A) \cap f^{-1}(B)$.

\begin{enumerate}
    \item Richtung: $f^{-1}(A \cap B) \subseteq f^{-1}(A) \cap f^{-1}(B)$
        \begin{enumerate}[label=$\circ$]
            \item Sei $x \in f^{-1}(A \cap B)$. Daraus folgt, dass $f(x) \in A \cap B$, also $f(x) \in A$ und $f(x) \in B$.
            \item Das bedeutet, dass $x \in f^{-1}(A)$ und $x \in f^{-1}(B)$, also $x \in f^{-1}(A) \cap f^{-1}(B)$.
        \end{enumerate}
    \item Richtung: $f^{-1}(A) \cap f^{-1}(B) \subseteq f^{-1}(A \cap B)$
        \begin{enumerate}[label=$\circ$]
            \item Sei $x \in f^{-1}(A) \cap f^{-1}(B)$. Dann gilt $x \in f^{-1}(A)$ und $x \in f^{-1}(B)$.
            \item Das bedeutet, dass $f(x) \in A$ und $f(x) \in B$, also $f(x) \in A \cap B$.
            \item Somit gilt $x \in f^{-1}(A \cap B)$.
        \end{enumerate}
\end{enumerate}

\noindent Da beide Inklusionen gezeigt wurden, folgt $f^{-1}(A \cap B) = f^{-1}(A) \cap f^{-1}(B)$. \\

\noindent\textbf{Ergebnis:} Die Aussage ist wahr.\\

\paragraph{(b)} Für alle $A, B \subseteq Y$ gilt $f^{-1}(A \cup B) = f^{-1}(A) \cup f^{-1}(B)$.

\begin{enumerate}
    \item Richtung: $f^{-1}(A \cup B) \subseteq f^{-1}(A) \cup f^{-1}(B)$
        \begin{enumerate}[label=$\circ$]
            \item Sei $x \in f^{-1}(A \cup B)$. Dann gilt $f(x) \in A \cup B$, also $f(x) \in A$ oder $f(x) \in B$.
            \item Daraus folgt $x \in f^{-1}(A)$ oder $x \in f^{-1}(B)$, also $x \in f^{-1}(A) \cup f^{-1}(B)$.
        \end{enumerate}
    \item Richtung: $f^{-1}(A) \cup f^{-1}(B) \subseteq f^{-1}(A \cup B)$
        \begin{enumerate}[label=$\circ$]
            \item Sei $x \in f^{-1}(A) \cup f^{-1}(B)$. Dann gilt $f(x) \in A$ oder $f(x) \in B$, also $f(x) \in A \cup B$.
            \item Das bedeutet $x \in f^{-1}(A \cup B)$.
        \end{enumerate}
\end{enumerate}

\noindent Da beide Inklusionen gezeigt wurden, folgt $f^{-1}(A \cup B) = f^{-1}(A) \cup f^{-1}(B)$. \\

\noindent\textbf{Ergebnis:} Die Aussage ist wahr.\\

\paragraph{(c)} Für alle $A, B \subseteq X$ gilt $f(A \cap B) = f(A) \cap f(B)$.

\begin{enumerate}
    \item Richtung: $f(A \cap B) \subseteq f(A) \cap f(B)$
        \begin{enumerate}[label=$\circ$]
            \item Sei $y \in f(A \cap B)$. Dann existiert ein $x \in A \cap B$, sodass $f(x) = y$. 
            \item Da $x \in A$ und $x \in B$, folgt $y = f(x) \in f(A)$ und $y = f(x) \in f(B)$.
            \item Also gilt $y \in f(A) \cap f(B)$.
        \end{enumerate}
    \item Richtung: $f(A) \cap f(B) \subseteq f(A \cap B)$
        \begin{enumerate}[label=$\circ$]
            \item Sei $y \in f(A) \cap f(B)$. Dann existieren $x_1 \in A$ und $x_2 \in B$, sodass $f(x_1) = y$ und $f(x_2) = y$.
            \item Da jedoch $x_1$ und $x_2$ nicht unbedingt gleich sind (da $f$ nicht injektiv sein muss), folgt nicht, dass $y \in f(A \cap B)$.
        \end{enumerate}
\end{enumerate}

\noindent Da die Umkehrung nicht gilt, ist $f(A \cap B) \subseteq f(A) \cap f(B)$, aber nicht umgekehrt. \\

\noindent\textbf{Ergebnis:} Die Aussage ist falsch.\\

\paragraph{(d)} Für alle $A, B \subseteq X$ gilt $f(A \cup B) = f(A) \cup f(B)$.

\begin{enumerate}
    \item Richtung: $f(A \cup B) \subseteq f(A) \cup f(B)$
        \begin{enumerate}[label=$\circ$]
            \item Sei $y \in f(A \cup B)$. Dann existiert ein $x \in A \cup B$, sodass $f(x) = y$.
            \item Da $x \in A$ oder $x \in B$, folgt $y = f(x) \in f(A)$ oder $y = f(x) \in f(B)$, also $y \in f(A) \cup f(B)$.
        \end{enumerate}
    \item Richtung: $f(A) \cup f(B) \subseteq f(A \cup B)$
        \begin{enumerate}[label=$\circ$]
            \item Sei $y \in f(A) \cup f(B)$. Dann existieren $x_1 \in A$ oder $x_2 \in B$, sodass $f(x_1) = y$ oder $f(x_2) = y$.
            \item In beiden Fällen gilt $x_1 \in A \cup B$ oder $x_2 \in A \cup B$, also $y \in f(A \cup B)$.
        \end{enumerate}
\end{enumerate}

\noindent Da beide Inklusionen gezeigt wurden, folgt $f(A \cup B) = f(A) \cup f(B)$. \\

\noindent\textbf{Ergebnis:} Die Aussage ist wahr.

\end{document}




